\documentclass[12pt]{article}

\usepackage[paperwidth=8.5in,paperheight=11in,top=.75in,left=.75in,right=.75in,bottom=.75in]{geometry} 


\usepackage{amsmath,mathtools,latexsym, ifthen, calc}
\vfuzz10pt % Don't report over-full v-boxes if over-edge is small
\hfuzz10pt % Don't report over-full h-boxes if over-edge is small
\usepackage{xspace}
\usepackage{abbrevs, nth}
\usepackage[nodisplayskipstretch]{setspace}
\usepackage[compress]{natbib}
\usepackage{txfonts}

\bibpunct{(}{)}{,}{a}{}{,}
\setcitestyle{numbers,square}

\defcitealias{crossleyhumberstone77}{Crossley-H'stone 1977}
\defcitealias{davieshumberstone81}{Davies-H'stone 1981}


%Murray-addd (editing)
\usepackage{color, soul}
\usepackage{comment}

%Place in your path: benj-numbering.sty, newbenj.bst, hmw.bib


\begin{document}

\subsection*{Research team, previous output, and student training}

\subsubsection*{A.\quad Research team}

Our research team comprises three Canadian philosophy professors: Benj Hellie
(Professor of philosophy at the University of Toronto), Adam Russell Murray
(Assistant Professor of philosophy at the University of Manitoba), and Jessica
M.\ Wilson (Professor of philosophy at the University of Toronto). Hellie is
grant applicant and primary investigator; Murray and Wilson are each
co-applicants.  \ul{It is anticipated that each member of the research team will
contribute a roughly equal proportion of the proposed research}. 

The RMM program in modal metaphysics has since its inception been a primarily
collaborative endeavour. The program's founding article (`Relativized
Metaphysical Modality'; Murray and Wilson 2012) developed out of a graduate
seminar taught by Wilson and attended by Murray. Wilson and Hellie would
subsequently supervise Murray's 2017 doctoral dissertation on RMM at the
University of Toronto (\emph{Perspectives on Modal Metaphysics}). More
recently, Hellie was the primary drafter of a 2020 article, co-authored with
Murray and Wilson, further elaborating
upon applications of RMM (`Relativized metaphysical modality: index and
context'). In light of these
collaborative interactions a team approach to the proposed research is very
natural.  

\ul{As primary investigator, Hellie will dedicate ...}. 
As co-applicant, Murray will dedicate 80\% of his research time to this
project. Murray will contribute to the authoring or co-authoring of several
papers in metaphysics and philosophical logic relevant to the project's main
themes, and to the anticipated monograph, as detailed in the Knowledge
Mobilization Plan. \ul{As co-applicant, Wilson will dedicate ...}. 


\subsubsection*{B.\quad Description of previous and ongoing research results}

\ul{\textbf{Hellie}'s previous and ongoing research results...}.  

\textbf{Murray}'s previous and ongoing research is mainly centered on
developing applications of RMM, primarily in metaphysics and in associated
areas of philosophical logic. 
In connection with the RMM program, Murray's previous research outputs include
the initial 2012 article co-authored with Wilson (`Relativized Metaphysical
Modality'), the more recent 2020 article co-authored with Hellie and
Wilson (`Relativized metaphysical modality: index and context'), of which
Hellie was the primary drafter, and the 2022 article `Propositional dependence
and perspectival shift'. The 2012 article develops the founding
insight of the RMM program in modal metaphysics in service of a novel solution
to `Chisholm's Paradox' of material origins essentialism, and a related
`undermining' puzzle arising for a necessitarian theory of laws.  The 2020 
article further develops RMM in connection with a cluster of modal puzzles,
concerning material origins, natural laws, and ontology, respectively, each
arising against the backdrop of a particular sub-fragment of a quantified
modal language (the language of quantified S5 with identity). Finally, the
2022 article elaborates
upon the significance of RMM for debates in `higher-order' modal metaphysics,
concerning the existence and nonexistence of singular (`object-involving')
properties and propositions, and sketches a possible worlds
semantics for higher-order modal logic capable of accomodating the core
doctrines of RMM. 

A secondary project in the philosophy of
language, recently completed, was the joint preparation (with Chris Tillman of
the University of Manitoba) of the \emph{Routledge Handbook of Propositions}
(Routledge UK; forthcoming 2022). The volume comprises 33 original articles by
an international team of scholars, and a substantial introductory chapter and
overview by
Murray and Tillman. The volume's focus is  on both historical and contemporary
theories of propositions, and on associated topics in logic, philosophy of
language, and philosophy of mind. 

Murray's primary ongoing research extends the applications of RMM to issues in
modal ontology: `Existence and
nonexistence in modal perspectivism' develops the RMM solution to the
first-order `Barcanite' puzzle discussed in the project description, while
`Higher-order dependence' extends that solution to analogous,
property-theoretic, Barcanite puzzles. 
A third ongoing RMM-related project, `Further considerations on
Chisholm's Paradox', extends the RMM treatment of `tolerance' puzzles to
a modal-soritical paradox of material origins, and to the `Four-worlds
Paradox' (due to Nathan Salm\'{o}n), each currently receiving renewed attention in the literature. 



\ul{\textbf{Wilson}'s previous and ongoing research results ...}.  



\subsubsection*{C.\quad Description of proposed student training strategies}

Our project will provide a wide range of training opportunities for graduate students
at two Canadian institutions, at both the PhD and Master's level. 

The proposed budget contains funds that will enable us to hire two doctoral
students at the University of Toronto, and two MA students at the University
of Manitoba, as research assistants. There are a number of very capable
graduate students in both programs who are interested in metaphysics,
philosophy of language, philosophical logic, and history of analytic
philosophy and who would be well-suited to work on this sort of project.  At
Toronto, Hellie and Wilson intend to hire one research assistant working
primarily in logic and language, and one working primarily in metaphysics,
each for the duration of the project term.  At Manitoba, Murray intends to
hire two MA students each year, with research interests primarily in language
and metaphysics. 

Our research assistants will work closely with us in surveying and analyzing
literature in each of the above areas of philosophy.  This work will involve
following up on references and composing detailed reviews of significant
papers. Since many of these papers will be technical in nature, over the
course of the project our students will develop the ability to make complex
ideas easily accessible, and to draw connections between ideas developed in
various sub-literatures.  Our research assistants will also compile master
bibliographic files for relevant sources identified in the course of their
investigations, and will be responsible for communicating the results of their
work not only to us, but also to each other (hence we envisage there being 
a collaborative aspect
to the student training component of our project as well). 

Our research assistants will also work closely with us in preparing articles
for submission to journals, and in the preparation of the anticipated book
manuscript. Additionally, we have budgeted a large amount of money for each
research assistant to travel with us to at least one professional conference
during every year of the project. These activities will provide our students
with opportunities to practice presenting their own work on topics related to
the proposed project. Accordingly, over the project term our research
assistants will receive important hands-on training in various research
methods (including data management and analysis), in research collaboration,
and in knowledge mobilization. These skills are essential to success in
graduate school, but they are also widely transferrable, and applicable
outside of academia. 


Finally, given the collaborative nature of our project, we hope that through
such joint research various joint papers will emerge: the RMM program already has a
strong track record of faculty--student collaboration (Murray was the
doctoral student of Wilson and Hellie; and the founding publication came out
of Wilson's seminar, attended by Murray).  Co-authoring would provide an
excellent way of helping our graduate students learn how to craft a publishable
piece of research. It would also help our students learn how to navigate the
publishing process. It is now almost required that a graduate student have
some publications in order to be successful in applying for tenure-track jobs.
Serving as a research assistant for this project would help in developing the
skills required for successfully publishing work during graduate school.



%Materials from Benj's original:

\begin{comment}

My research this decade has ranged relatively widely over various subfields of
philosophy, addressing questions in perceptual epistemology and phenomenology,
metaphysics of perspective, self-reference, `metapsychology' (the `psychology
of psychology'---what is involved when we think about the mind), practical
reasoning theory, the semantics of imperatives, the semantics of deontic
modality, the `history of David Lewis', mental causation, the post-1945
history of the mind--body problem, the philosophical and technical foundations
of logical theory and theory of meaning, the mind--body problem---as well as a
collaborative project in modal logic--metaphysics, with my departmental
colleague Jessica M.\ Wilson and our former co-PhD advisee, Adam Russell
Murray (philosophy, Manitoba): this last initially a side project while I
completed work on a long-running program in philosophy of mind, now moving to
the fore as the issue for the proposed research.

Is there anything to unify it all---in particular, to unify the earlier work's
focus on philosophy of mind with the metaphysical focus of the proposed
research? Perhaps surprisingly, yes. In the background of both the earlier
work on \emph{mentalism} and the current work on \emph{RMM} is a worry that
our philosophical era has too often been significantly off base in its
treatment of \emph{perspective}, and that progress on many issues will require
setting our understanding aright. The differences in the subject-matter
addressed in the mentalism and RMM projects ramify to distinctions between the
relevant varieties of perspective; but their common invocation of perspective
(in the abstract) makes for a continuity of method: namely, each project makes
a fundamental appeal to broadly `two-dimensional' (2D) semantical apparatus in
the representation of perspective---with the details differing between the
cases, and in each case offering significant novelty vis-a-vis standing
proposals in the literature.

As noted in the `Knowledge mobilization' document, I submitted a few months
ago a contracted book MS to Oxford UP-USA, titled \emph{Out of This World:
Logical Mentalism and the Philosophy of Mind} (`OOTW') (I expect reports
toward the end of this calendar year): developing this book occupied much of
my research time for about ten years. In the mentalism project discussed in
OOTW, the relevant variety of `perspective' is that of the individual state of
conscious mentality. The endeavour there is to treat this perspective as the
explanatory bedrock---in contrast, the book contends, to the tradition of
analytic philosophy of mind, which seeks bedrock instead in the totality of
facts, or `\emph{the (actual) world}': doing so lifts from the actual world
the metaphysical burden of supporting the existence of mentality, thereby
whisking away worries about the metaphysics of mind (foremost, the mind--body
problem); this in turn undermines an epistemological hypothesis assimilating
knowledge of the mental to knowledge of the world (suggesting instead a
competing hypothesis that the mental is by its nature self-knowing). The
hypothesis about `explanatory bedrock' is cashed out with a view of
\emph{logic} (as good a candidate for bedrock as any) as concerned not with
\emph{truth} (as bestowed by the actual world and its ilk, the possible
worlds) but rather \emph{endorsement} (as bestowed by my present conscious
mental state and its ilk, the intelligible states of consciousness). Replying
to the worry that this `logical mentalism' collapses into an unattractive
psychologism, I defend a strongly `rationalistic' approach in philosophical
psychology, in which mental states are representable (at a certain level of
grain) by mathematically simple entities (along the lines of the set-of-worlds
proposition). Logical consequence can then be analyzed along relatively
familiar two-dimensionalist lines, but with a twist: as preservation of
`designation' at all diagonal points, but with `designation' as
\emph{endorsement} and points as mental states. This enables the
`self-knowingness by nature' of mental states to be represented with an
analogue to the familiar \emph{actuality} operator, while off-diagonal points
serve as candidates for use in simulating other subjects.

(OOTW folds in and improves upon material from a number of my other most
significant career research contributions. Keyed to the numbering system in
`Research contributions' subhead~3: contributions~2 and~6 sketch the technical
background to logical mentalism, while contribution~2 offers an empirical case
for that position; contribution~6 applies that apparatus to the dissolution of
the mind--body problem, while contribution~4 applies the associated doctrine
to the dissolution of questions about mental causation; contributions~2, 4, 5,
and~6 develop aspects of the `rationalistic' approach; and contribution~5
develops the epistemology of mind available to the mentalist.)

The use made of two-dimensionalism in the RMM project is significantly more
conservative than the radical application in the mentalism project. In the RMM
project, the relevant notion of `perspective' is a more `situational' notion,
closer in spirit to the conception explored in the traditional (`logical
realist') two-dimensionalism of Lewis, Kaplan, and others. The result is more
conservative logically than is the doctrine of the mentalist project---logic
is still concerned with truth, and indeed a motivating spirit is simplicity
and conservatism of logical systems and their semantic analysis---and, where
the mentalism project seeks to whisk away metaphysical questions about the
mind, the RMM project is instead concerned to face metaphysical questions
squarely and address them at face value. 

Despite this marked contrast of spirit, the stances of the mentalism and RMM
projects are to my mind mutually coherent. A central emphasis of the mentalism
project is that the traditional `realist' logic of truth is available within
the mentalist logic of endorsement, with the former recoverable by imposing
expressive or model-theoretic constraints on mentalist languages or
interpretive frameworks. The mentalist is not opposed to the use of
possible-world--like entities in the analysis of consequence (as bestowers of
`designation'); rather, it is the realist who slams the door here, excluding
all other entities from this role. Indeed, the mentalist acknowledges that we
often reason about `objective', non-mental matters: for this purpose, use of a
logic of truth (perhaps with worlds as `godlike mental states'?)\ is
appropriate; and, moreover, it is exactly in such cases in which questions of
metaphysics naturally arise---the mentalist's complaint is with extending this
style of reasoning beyond its natural home, to reasoning about the mental.
Questions for the RMM project arise in reasoning about the non-mental, and
thus do not run afoul of these mentalist scruples.

(In an interesting irony, the OOTW and RMM projects can be seen collectively
as teasing apart the philosophical aims and technical apparatus of
epistemological two-dimensionalism (E2D): the RMM project invokes a truth- and
worlds-based logical apparatus familiar from E2D, but in pursuit of issues in
metaphysics rather than epistemology and philosophy of mind; while the OOTW
project shares with E2D the pursuit of issues in epistemology and philosophy
of mind, but contrastingly invokes an endorsement- and mental states-based
logical apparatus. E2D therefore stands accused of misfitting its apparatus
and its philosophical ambitions: when each is fit instead with its appropriate
partner, the result is bifurcation into my pairing of OOTW and RMM.)

The \emph{historical} papers (`David Lewis and the Kangaroo'; `An
analytic--hermeneutic history of `Consciousness'\,') also touch on the
philosophical significance of perspective. Each canvases the appearance and
evolution of certain tensions in a prominent, recent philosophical
corpus---Lewis's work; the main stream of work on `metaphysics of mind' (Ryle,
Smart, Lewis, Kripke, Block, Shoemaker, Nagel, Jackson, Dennett, Harman,
Chalmers, and Horgan and Tienson). `Kangaroo' takes as its explanandum a
grave, surprising mistake (Arntzenius) about `de se attitudes';
`Consciousness' traces the evolving, unstable construction of the putative (to
my mind, specious) notion of `phenomenality'. Responsibility goes in both
cases to the doctrines attacked in the mentalist project: the `realist' logic
of truth, the use of the actual world as explanatory bedrock, attendant
inabilities to theorize adequately about the perspective of consciousness.

I should note points in past or ongoing work which bear on the development of
the proposed papers on philosophy of language (see the `Knowledge
mobilization' document). 

`Why and whither' is a polemical discussion of a technical framework doing
ineliminable work in the submitted MS: the overall point there---the core
argument, as I understand it, of Lewis's titanic 1980 `Index, context, and
content' (ICC)---is this. If the semantic values of a language have an
argument-place free, which is nevertheless saturated in the contents it
expresses, that argument-place must be saturated in context, so semantic
values cannot be identified with contents. If, moreover, such an
argument-place is available for rigidification against intensional operators,
then that argument-place is capable only of saturation by context once
rigidified, but is capable of binding by operators prior to rigidification,
even though the parametric values (time, place, world) in its domain are the
same: accordingly, we must recognize a \emph{functional} distinction (and not
a merely metaphysical distinction) among types of argument-place, between the
unshiftable `contextual' and the shiftable `indexical'. But then if indexical
arguments are to be saturated by context following semantic composition, they
must be converted to contextual arguments---and this happens perforce not
during semantic composition, but only afterward: \emph{postsemantically}.
(OOTW invokes both indexical and contextual occurrences of a `mental state'
parameter---the former to handle iterated psychological reports, the latter to
handle the `essential self-knowledge' of a pure first-person perspective.)

`Triviality', as noted, invokes a high-powered application of postsemantics.
An interest in conditionals is a natural concommitant to an interest in
modals; and, in light of the manifest virtues of Stalnaker/Lewis style
semantics for the conditional, any approach urging some sort of logical or
semantical radicalism inevitably faces the question of how it might do quite
as well. The puzzles addressed in this paper are also ones that pose evident
threats of different sorts to broadly mainstream, two-dimensional,
compositional semantics: Williamson triviality problematizes how natural
language indicatives can (with appearances) encode a `deduction theorem' for
2D-style entailments (how $P \vdash Q$ can require $\vdash P \lcon Q$), if an
\emph{if}-clause is a nonmonstrous index-shifting operator; while Gibbard
triviality problematizes how the order of stacked \emph{if}-clauses can
(again, with appearances) be logically unimportant, if an \emph{if}-clause is
a `local' (it acts only on its operand and then vanishes) index-shifting
operator. The answer: an \emph{if}-clause isn't an index-shifting operator!
With only one \emph{if}-clause and without 2D complications, the Stalnaker
index-shifting analysis is correct; but stacked \emph{if}-clauses seem to
require some kind of delayed shift; and a deduction theorem seems to require
some kind of context shift. The proposed analysis adapts the Stalnaker
approach to yield both.

`Monstrosity' addresses an age-old question: Kaplan does not defend the ban on
monsters, so why insist on it? The Williamsonesque, `slingshot'-style argument
occurred to me while working on the conditionals paper (perhaps earlier
historical investigations---into the Church--Carnap exchange---were in the
back of my mind).

`Apollonian' proposes in part to significantly expand on cursory remarks in
the 2020 RMM survey article. The large amount of time I spent thinking about
Lewis's `ICC' and the place of its subject-matter in the development of the
Lewis corpus imbued a sensitivity to neglect of its considerations---to which
Lewis himself was subject, by his own acknowledgement there, in his earlier
work---as well as an appreciation of the tremendous mental acrobatics required
for Lewis to eventually resolve the grave difficulties addressed. Reading work
from the early development of modal metaphysics, this neglect jumps off the
page---but then also comes as no surprise. In the course of working up
`Kangaroo', I engaged an extensive survey of Lewis's corpus (with a
quantitative component, to represent relations of dependence and mutual
relevance of doctrine among his publications), only a small part of which went
into that paper; and have taught undergrad and grad seminars on this material
in four semesters. So I am well-prepared to do the \emph{Lewis} aspect of this
paper (I have also been reading through the recently published volumes of
letters, and hope to more fully incorporate their revelations into my
investigations before too long). I have taken less of a scholarly interest in
Kripke; still, he has published less than Lewis; and while his work also
raises enigmas of a different variety than those from Lewis's, they are more
widely discussed: accordingly, the \emph{Kripke} aspect of the paper may be
relatively straightforward to develop.

\end{comment}


\end{document}





