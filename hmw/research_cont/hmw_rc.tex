\documentclass[12pt]{article}

\usepackage[paperwidth=8.5in,paperheight=11in,top=.75in,left=.75in,right=.75in,bottom=.75in]{geometry} 

\usepackage{benj,abbrevs}

\usepackage{txfonts}

\begin{document}

\subsection*{Research contributions}

\subsubsection*{1.\ \quad Relevant research contributions over the last six years}

\paragraph{Refereed contributions}
\begin{itemize}
	\item[] `Relativized metaphysical modality: Index and context', with Adam Russell Murray and Jessica M.\ Wilson, in Ot\'avio Bueno and Scott Shalikowski, editors, \emph{The Routledge Handbook of Modality}, pp.~82--99. London: Routledge, 2020.
	
	\item[] `An analytic-hermeneutic history of Consciousness', in Kelly Michael Becker and Iain Thomson, editors, \emph{The Cambridge History of Philosophy, 1945 to 2010}, 74--81. Cambridge: Cambridge University Press, 2019.
	
	\item[] `Semantic gaps and protosemantics', in Acacio de Barros and Carlos Montemayor, editors, \emph{Quanta and Mind: Essays on the Connection between Quantum Mechanics and Consciousness}, pp.~201--221. Berlin: Springer, 2019.
	
	\item[] `Praxeology, imperatives, and shifts of view', in Rowland Stout, editor, \emph{Process, Action, and Experience}, pp.~185--209. Oxford: Oxford University Press, 2018. 
	
	\item[] `David Lewis and the Kangaroo: Graphing philosophical progress', in Russell Blackford and Damien Broderick, editors, \emph{Philosophy's Future: The Problem of Philosophical Progress}, pp.~213--225. New York: Blackwell, 2017. 
	
	\item[] `Rationalization and the Ross Paradox', in Nate Charlow and Matthew Chrisman, editors, \emph{Epistemic Modality}, pp.~283--323. Oxford: Oxford University Press, 2016.
	
	\item[] `Obligation and aspect', \emph{Inquiry} 59:398--449, 2016.
	
	% \item[*] `Love in the time of cholera', in Berit Brogaard, editor, \emph{Does Perception Have Content?}, pp.~242--61. Oxford: Oxford University Press, 2014.
	
	% \item[*] `It's still there!', and `Yep---still there', in Richard Brown, editor, \emph{Consciousness Inside and Out}, pp.~127--36 and 163--9. Berlin: Springer-Verlag, 2014.
	
	% \item[*] `Against egalitarianism', \emph{Analysis} 73:304--20, 2013.
	
	% \item[*] `The multidisjunctive conception of hallucination', in Fiona MacPherson and Dimitris Platchias, editors, \emph{Hallucination}, pp.~149--73. Cambridge, MA: The MIT Press, 2013.
\end{itemize}


\paragraph{Non-refereed contributions (all academic talks or commentaries)}
\begin{itemize}
	
	\item[2021] Consciousness does not have a metaphysics, Oxford Metaphysics of Consciousness Seminar Series, Faculty of Philosophy, University of Oxford


	Manufacturing defects, with Jessica Wilson: Invited Symposium on Pseudoproblems in Metaphysics (co-symposiasts, Amie Thomasson and LA Paul), APA Central Division
	
	\item[2020] Endorsement logic and the new deflationism: co-keynote address, \emph{Is Metaphysics Indispensible?}\ workshop, Department of Logic and Theoretical Philosophy, Complutense University of Madrid, Spain (postponed due to COVID-19)

	Dissolving the mind--body problem by repairing logic: International Speaker Series, Department of Philosophy, National Research University, Higher School of Economics, Moscow, Russia (postponed due to COVID-19)

	Mentalist logic of belief reports: Department of Philosophy, University of Gothenburg, Sweden (postponed due to COVID-19)
	
	\item[2019] Emergence and metapsychological expressivism: \emph{Rethinking Emergence} conference, University of Lisbon, Portugal
	
	Semantic gaps and protosemantics: (i) \emph{Russell} workshop, Healdsburg, CA; (ii) \emph{Mind and Action} conference, Department of Philosophy, Shandong University, Jinan, China; (iii) \emph{Ranch Metaphysics} workshop, White Stallion Ranch, Tucson, AZ, with comments by John Bengson (Wisconsin--Madison)
	
	Comment on Daniel Stoljar, Pessimism about philosophical progress---Why is it so widespread?: \emph{Philosophical Progress} workshop, ConceptLab and University of Tokyo, Tokyo
	
	Comment on David Boylan, Putting `ought's together: APA Pacific Division, Vancouver
	
	\item[2018] Has analytic philosophy \emph{created} the `hard problem of consciousness'?: Seminario de Investigadores, Instituto de Investigaciones Filos\'oficas, UNAM, CDMX, Mexico 
	
	Reasoning about conditionals and conditional reasoning: Seminario de Filosof\'ia del Lenguaje, Instituto de Investigaciones Filos\'oficas, UNAM, CDMX, Mexico
	
	
	\item[2017] The semantic defectiveness of `Grounding' and `Consciousness', with Jessica Wilson: (i) \emph{Grounding and Consciousness} workshop/conference, Department of Philosophy, NYU, La Pietra, Florence, Italy, with comments by Catharine Diehl (Humboldt) and Lisa Vogt (Barcelona/LOGOS); (ii) Canadian Philosophical Association, Toronto

	From externalism to expressivism: \emph{CaSE: Consciousness and Semantic Externalism} workshop, Department of Philosophy, NYU

	Benj Hellie's `There it is': respondant at discussion session, led by Philipp Blum (Lucerne), at \emph{Cogito: Yes or No?}\ workshop/conference, Department of Philosophy, University of Geneva, Ligerz, Switzerland

	Endorsement-logic, simulationism, and the \emph{cogito}: \emph{Cogito: Yes or No?}\ workshop/conference, Department of Philosophy, University of Geneva, Ligerz, Switzerland
	
	\item[2016] Deconstructing intensions: \emph{Wilson--Hellie} workshop, Department of Philosophy, University of Edinburgh, Scotland

	Above the verb with endorsement theory: \emph{Aspect and Modality} workshop, Departments of Philosophy and Linguistics, University of Western Michigan, Lansing, MI
	
	% \item[2015] Ross-Paradoxical anankastic conditionals: Department of Philosophy, University of Edinburgh, Scotland
	
	% \item[2014] Out of this world: (i) Department of Philosophy, University of Alberta; (ii) \emph{Metaphysics of Mind} workshop, Department of Philosophy, University of Edinburgh, Scotland
	
	% Here and there (Part I: Here): (i) Brian McLaughlin and Susanna Schellenberg's seminar on spatial perception, Department of Philosophy, Rutgers University; (ii) \emph{Oriented Worlds Ramble}, Berkeley, CA
	
	% How we do: Department of Philosophy, Bo\u gazi\c ci University, Istanbul, Turkey
	
	% Knowing what it is like to converse in $L$: \emph{Arizona Ontology Conference},  Department of Philosophy, University of North Carolina, White Stallion Ranch, Tucson, AZ, with comments by Jack Spencer (MIT)
	
	% \item[2013] On the creation of the One, the It, the World, the Self, and God: Metaphysics Group, Arch\'{e}, University of St Andrews, Scotland
	
	% Three grades of context-dependence: Propositions, Tense, and Indexicality Group, Arch\'{e}, University of St Andrews, Scotland
	
	% Why isn't justified true belief knowledge?: (i) Department of Philosophy, CUNY-Graduate Center; (ii) \emph{Epistemology and Metaphysics} workshop, Department of Philosophy, University of Edinburgh, Scotland; (iii) Evidence, Justification, and Knowledge Group, Arch\'{e}, University of St Andrews, Scotland
	
	% Indeterminacy, knowledge, and contraposition: Centre for the Study of Mind and Nature, Department of Philosophy, University of Oslo, Norway
	
	% Out of this world: (i) \emph{Phenomenal Concepts},  Department of Philosophy, Federal University of Rio de Janeiro, Brazil, with comments by Wilson Pessoa Mendon\c ca (UFRJ); (ii) Invited Paper on the Metaphysics of Subjectivity, APA Pacific Division, San Francisco, with comments by Brie Gertler (UVA) and Geoff Lee (Berkeley)


	
	

\end{itemize}


\paragraph{Forthcoming contributions}
\begin{itemize}
	\item[] \emph{Out of This World: Logical Mentalism and the Philosophy of Mind}, submitted for review to New York: Oxford University Press (under contract).
	

\end{itemize}

% \subsubsection*{2.\ \quad Other research contributions over the last six years}


\subsubsection*{3.\ \quad Most significant career research contributions}

\begin{enumerate}
	\item \emph{Out of This World: Logical Mentalism and the Philosophy of Mind}, submitted for review to New York: Oxford University Press (under contract).
	
	`Logical mentalism' is the doctrine that what is preserved under a logically valid argument is not \emph{truth} (a nonmental, objective matter) but rather \emph{endorsement} (a mental, subjective matter). The book explores the doctrine in various aspects: technical, foundational, and in potential application to the philosophy of mind. On the technical side, I propose to interpret mentalism by unifying familiar `two-dimensionalist' apparatus with a novel, broadly algebraic approach to `points of evaluation', and propose a certain sort of effect with \emph{negation} as characteristically mental. On the foundational side, I note that mentalism is without allure absent a strongly rationalist conception of mentality, and explore in some detail a `simulationist' theory apt to secure the conception. On the applied side, I offer a semantics for mental reports within logical mentalism adequate to this simulationism, and then discuss how the resulting package dispells various long-standing vexations in the metaphysics and epistemology of mind. (This work adapts, updates, and synthesizes material from contributions (2) and (4)--(6).)
	
	
	\item `Rationalization and the Ross Paradox', in Nate Charlow and Matthew Chrisman, editors, \emph{Epistemic Modality}, pp.~283--323. Oxford: Oxford University Press, 2016.
	
	The `Ross Paradox' is the failure of `weakening' for imperatives: `pay the rent!'\ does not entail `pay the rent or drink up my wine!'. I postulate a novel kind of content for imperatives, the `granulated proposition' consisting of a partition together with a proposition fusing some of its cells, to explain this logical oddity. I then move `backward' and `forward': `backward', to explain what it is about practical reason that requires contents of \emph{intention} to involve the partition, and what it is about theoretical reason that does not require this of the contents of \emph{belief}; and `forward', to explain other manifestations of Ross-paradoxicality: in the speech act of \emph{command}, in `obligative' uses of \emph{must}, and in avowals of intention.
	
	\item `Relativized metaphysical modality: Index and context', with Adam Russell Murray and Jessica M.\ Wilson, in Ot\'avio Bueno and Scott Shalikowski, editors, \emph{The Routledge Handbook of Modality}, pp.~82--99. London: Routledge, 2020. 
	
	We consider three sorts of puzzle for the `simplest, strongest', S5/Barcanite logic of modality, individuation, and existence: an \emph{undermining} puzzle involving laws of nature for the propositional subfragment; \emph{Chisholm's paradox} of moderate material origin essentialism for the referential subfragment; and apparent \emph{contingent existence/nonexistence} as a challenge to the full quantificational fragment. These puzzles have long seemed intractable, but we argue that a critical technical resource, otherwise needed, has been inadequately exploited---namely, `context-sensitivity'. With this (broadly `two-dimensionalist') resource in hand, our metaphysics and our logic need not clash.
	
	\item `Praxeology, imperatives, and shifts of view', in Rowland Stout, editor, \emph{Process, Action, and Experience}, pp.~185--209. Oxford: Oxford University Press, 2018.
	
	Michael Thompson claims that the basic linguistic form of practical reason is `I am $A$-ing because I am $B$-ing'; recent discussion has drawn on this claim to support a `processive' metaphysics of action to compete with Davidsonian `event-causal' metaphysics. But more basic than this is `bare imperatival implication': `$A$! ---so, $B$!'. But a `bare imperative' does not characterize anything about the world, so there can be no metaphysics of action. Instead, these bare imperatives are `radically first-personal'; the role of practical reason in \emph{explaining} agentive behavior essentially requires a `shift of view' between the radical first person and the `objective' third person, mediated by the direct-realist `perceptual given'.
	
	\item (a) `There it is', \emph{Philosophical Issues} 21:110--64, 2011; and sequels: (b) `It's still there!', and `Yep---still there', in Richard Brown, editor, \emph{Consciousness Inside and Out}, pp.~127--36 and 163--9. Berlin: Springer-Verlag, 2014; (c) `Love in the time of cholera', in Berit Brogaard, editor, \emph{Does Perception Have Content?}, pp.~242--61. Oxford: Oxford University Press, 2014.
	
	These papers consider `direct realist' philosophies of perception, understood so that perceptual content is \emph{factive} and \emph{external-world concerning}. First: without \emph{factivity} perception cannot do its work in rationalizing belief and `anchoring' agency to the body of the agent; and there is no hope of resolving attendant difficulties arising from `hallucination' by retreating from \emph{externality}. Second: such cases show that sometimes, one's picture of the world is incoherent, involving mutual incompatibility of perceptually given facts with their rationally downstream `interpretation': the difficulty here is that this is never part of how one finds one's mental state to be, but is only noticed by external interpreters---discomfitingly, if the goal of the external interpreter is to characterize how one finds one's mental state to be; this lends support to an metapsychological expressivism, on which the apparent clash need not be resolved.
	
	\item `Semantic gaps and protosemantics', in Acacio de Barros and Carlos Montemayor, editors, \emph{Quanta and Mind: Essays on the Connection between Quantum Mechanics and Consciousness}, pp.~201--221. Berlin: Springer, 2019.
	
	This paper exploits the post-Kripkean, `algebraic' resource of propositional points of evaluation, as embedded in a classical two-dimensionalist framework, on behalf of metapsychological expressivism, and uses the resulting analysis to analyze various putative `semantic gaps' between the mental and physical, and then to interrupt attendant arguments for mind--body dualism.
	
	
	
	
	
\end{enumerate}

\subsubsection*{4.\ \quad Career interruptions and special circumstances}

None

\subsubsection*{5.\ \quad Contributions to training}

\begin{itemize}
	\item[] At the University of Toronto over the last six years, I have been involved in a number of PhD committees, some in a supervisory role, others as a member of the committee.
	
	From 2018, I have chaired the committee of Zach Weinstein. The project, in aesthetics, takes off from the approach in `rational psychology' outlined in my `Rationalization and the Ross Paradox', and is concerned with the `enigmatic' character of modern artwork: its elusiveness in regard to complete or determinate characterizations of meaning, and the manner of aesthetic value this conveys to its audience.
	
	From 2015, I served as the co-supervisor (with Jessica M.\ Wilson) of the dissertation by Adam Russell Murray, successfully defended in 2017: Murray has been employed since 2018 as an Assistant Professor (tenure track) at University of Manitoba. This supervision resulted in the co-authored paper (3), above; developing the program it surveys is the aim of the proposed research for the granting period.
	
	From 2019 and 2018, respectively, I have sat on the thesis committees of two students: Eliran Haziza, who is working on the semantics and pragmatics of questions, and Andrew Lavigne, who is working on the philosophy of linguistics; and in AY 2021--22, I sit on the qualifying committee of Dwight Crowell, who is working on the metaphysics of `Russellian Monism'.
	
	From 2012, I sat on the thesis committee of Dan Rabinoff, supervised by Jessica M.\ Wilson, who defended his thesis on the logic and metaphysics of relative identity in 2020: Rabinoff has found employment as an `ontological engineer'.
	
	From 2010 to the 2015 defense, I sat on the thesis committee of Luke Roelofs, co-supervised by Bill Seager and Jessica M.~Wilson: Roelofs, who wrote on the `combination problem' for panpsychism, has converted the thesis into a published book (Oxford UP), and is now on a post-doc at NYU, following a first post-doc at ANU and a second at Bonn.
	

	
	
	
	
\end{itemize}



\end{document}

