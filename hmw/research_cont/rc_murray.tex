\documentclass[12pt]{article}

\usepackage[paperwidth=8.5in,paperheight=11in,top=.75in,left=.75in,right=.75in,bottom=.75in]{geometry} 


\usepackage{amsmath,mathtools,latexsym, ifthen, calc}
\vfuzz10pt % Don't report over-full v-boxes if over-edge is small
\hfuzz10pt % Don't report over-full h-boxes if over-edge is small
\usepackage{xspace}
\usepackage{abbrevs, nth}
\usepackage[nodisplayskipstretch]{setspace}
\usepackage[compress]{natbib}
\usepackage{txfonts}

\bibpunct{(}{)}{,}{a}{}{,}
\setcitestyle{numbers,square}

\defcitealias{crossleyhumberstone77}{Crossley-H'stone 1977}
\defcitealias{davieshumberstone81}{Davies-H'stone 1981}


%Murray-addd (editing)
\usepackage{color, soul}
\usepackage{comment}


\begin{document}


\subsection*{Research contributions}

\subsubsection*{1.\ \quad Relevant research contributions over the last six years}

\paragraph{Refereed contributions}

\begin{itemize}
	\item[] \emph{The Routledge Handbook of Propositions}, co-edited with
Chris Tillman. London: Routledge 2022.  
	\item[] `Propositional Dependence and Perspectival Shift', in Chris
Tillman and Adam Russell Murray, editors, \emph{The Routledge Handbook of
Propositions}, pp.\ 393-407. London: Routledge 2022.  
	\item[] `Relativized Metaphysical Modality: Index and Context', with
Benj Hellie and Jessica M.\ Wilson, in Ot\'avio Bueno and Scott Shalkowski,
editors, \emph{The Routledge Handbook of Modality}, pp.~82--99. London:
Routledge, 2020. (Note: Hellie was the primary drafter of this article; all
other co-authored publications are equal.)
	
\end{itemize}


\paragraph{Non-refereed contributions (academic talks and commentaries)}
\begin{itemize}

\item[{2022}] Barcan formulas and the limits of possibility: University
of Lisbon (Language, Mind and Cognition research group)

\item[{\phantom{}}] Selection and propositional structure: with Chris
Tillman, Sixth Italian Conference on Analytic Metaphysics and Ontology, 
L'Aquila, Italy 

\item[2020] Being constraints in modal metaphysics: University of Barcelona
(Logos Research Center in Analytic Philosophy) (postponed due to COVID-19)

\item[{2019}] Modality and propositional dependence: Analytic Philosophy
Workshop, Yonsei University, Seoul, South Korea 

\item[{\phantom{}}] Propositional dependence and perspectival shift}: Birmingham Higher-Order
Metaphysics Workshop, Birmingham University, Birmingham, U.K. 

\item[{\phantom{}}] N$\nrightarrow$E: with Chris Tillman, University of Nevada, Las Vegas
(department colloquium) 

\item[{\phantom{}}] Symposium talk on modal paradox: American Philosophical
Association, Pacific Division (invited)

\item[{\phantom}] Symposium talk on propositional dependence: Canadian
Philosophical Association, University of British Columbia, Vancouver, Canada
(invited)

\item[{\phantom{}}] Comments on Juvshik, Against a neo-Quinean metaontology: Canadian
Philosophical Association, University of British Columbia

\item[{\phantom{}}]  Comments on Ganson, Causally idle objects of perception: American
Philosophical Association, Central Division, Denver CO. 

\item[{\phantom{}}]  Comments on LaCroix, Reference by proxy and truth-in-a-model: Western
Canadian Philosophical Association, University of Calgary

\item[{2018}] Eternal existents: Canadian Philosophical Association,
Universit\'e du Qu\'ebec \`a Montr\'eal, Montr\'eal, Canada 

\item[{\phantom}] Existence and nonexistence in modal perspectivism,
Society for Exact Philosophy, University of Connecticut

\item[{\phantom}] Tolerance puzzles in modal metaphysics: Ko\c c University,
Istanbul, Turkey (department colloquium; invited)

\item[{\phantom}] Chisholm's Paradox: The University of Manitoba
(department colloquium; invited)

\item[{\phantom}]  Comments on Woods, Many, but \st{almost} one: Canadian Philosophical
	Association, Universit\'e du Qu\'ebec \`a Montr\'eal  


\item [{2017}]  Dependence in the necessary framework of objects:
Canadian Philosophical Association, Ryerson University, Toronto, Canada  

\item [{2016}] Modal dependence: Bilkent University, Ankara, Turkey (Department
colloquium; invited) 

\end{itemize}

\paragraph{2.\ \quad Other research contributions}

\begin{itemize}

\item[] Dissertation: \emph{Perspectives on Modal Metaphysics}; Supervisors:
Benj Hellie and Jessica Wilson; Readers: Nate Charlow and Nick Stang; External
Examiner: Louis deRosset. University of Toronto, 2017.

\end{itemize}


\subsubsection*{3.\ \quad Most significant career research contributions}
	
\begin{enumerate}	

\item `Relativized Metaphysical Modality', with Jessica Wilson, in Karen Bennett
and Dean Zimmerman, editors, \emph{Oxford Studies in Metaphysics} 7, 189--226. 2012. 

\noindent This article develops the founding insight of the RMM program in
modal metaphysics, according to which matters of metaphysical possibility and
necessity are best understood as relativized to a `modal perspective' (a
possibility considered as-actual). The article develops that insight in
service of a novel solution to `Chisholm's Paradox' of material origins
essentialism, and to a related `undermining' puzzle arising for
a robust necessitarian theory of laws.  

\item 'Relativized Metaphysical Modality: Index and Context', with Benj Hellie
and Jessica M.\ Wilson, in Ot\'avio Bueno and Scott Shalkowski, editors,
\emph{The Routledge Handbook of Modality}, pp.~82--99. London: Routledge.
2020.

This article further develops RMM in connection with 
a cluster of modal puzzles, concerning material origins, natural laws, and
ontology, respectively, 
each arising against the backdrop of a particular
sub-fragment of a quantified modal language (the language of quantified S5
with identity). These puzzles have long been viewed as reflecting a
fundamental tension between intuitive metaphysics and simple 
(quantified) modal logic and semantics. But we show how the puzzles all arise
on the basis of a systematic conflation between as-actual and
as-counterfactual modal perspective. Incorporating this basic
two-dimensionalist insight into modal theorizing reconciles 
simple modal logic with intuitive metaphysics.  

\item `Propositional Dependence and Perspectival Shift', in Chris
Tillman and Adam Russell Murray, editors, \emph{The Routledge Handbook of
Propositions}, pp.\ 393-407. London: Routledge. 2022.  

\noindent This article elaborates upon the significance of RMM for 
debates in `higher-order' metaphysics concerning the existence and
nonexistence of 
singular (`object-involving') properties and propositions. Here, a
contingentist view is called upon by the apparent ontological dependence of
singular propositions upon their individual subject-matters, whereas a  
necessitist view is supported by more general and systematic considerations pertaining to the 
metaphysics of predication and higher-order quantification. But the tension is
merely apparent, and arises only on the basis of a strictly as-counterfactual
perspective on the space of metaphysical possibility. I show instead how intuitions 
supporting the dependence idea can be preserved in a higher-order necessitist
setting by representing the space of possible worlds (and subsets thereof) as
relativized to an as-actual perspective. 

\end{enumerate}

\subsubsection*{4.\ \quad Career interruptions and special circumstances}

My productivity slowed temporarily during 2020, due to the COVID-19 pandemic
and certain family-related difficulties.


\subsubsection*{5.\ \quad Contributions to training}

\paragraph{Research supervision}

I have supervised or co-supervised the work of the following students: 

\begin{itemize}

\item[] Greg Glatz (co-supervisor, current): MA thesis on debunking arguments
in epistemology

\item[] Ursula Chojko-Bolec (supervisor, completed): MA research project on practical
rationality; now Law at University of Alberta  

\item[] Harley Dyck (primary supervisor, completed): MA thesis project on aesthetics and the theory of
symbolic forms; now professionally employed 

\item[] Julia Minarik (examiner (MA thesis); earlier co-supervisor,
completed): MA research project on the metaphysics and aesthetics of tattoos; now PhD
student at Toronto  

\item[]  Tumpa Bithy (co-supervisor, completed): MA thesis on Russelianism and the problem
of empty names; now professionally employed

\item[] Lucas Bennett (supervisor, at Toronto, completed): undergraduate
research project on the ontology of fiction and semantics of fictional names;
now data scientist with Deloitte 


\end{itemize}
 
\paragraph{Other contributions to training}

\begin{itemize}

\item[] Acting graduate chair and general academic supervisor: University of
Manitoba Department of Philosophy; July 2019--January 2021

\item[] Faculty assessor: University of Manitoba Undergraduate Research
Awards; Fall 
2018--2020

\item[] Co-organizer (with Belinda Piercy): University of Toronto Undergraduate
Research Conference; Spring 2017 and 2018

\item[] Department teacher-training coordinator: University of Toronto
Graduate Department of Philosophy (for first-time instructors, teaching
assistants, and graders); Fall and Spring semesters 2016--2018

\end{itemize}


\end{document}

