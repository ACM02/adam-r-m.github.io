
The primary aim of the proposed research is to contribute to the exploration
of the ‘Relativized Metaphysical Modality’ (RMM) program. The central doctrine
of RMM is that ‘modal perspective’ is fundamentally involved in a variety of
phenomena of central interest to analytic metaphysics: namely, phenomena for
which intuitition supports a metaphysics of ‘bounded naturalism’, such that
the limits on metaphysical possibility are somehow ‘generated’ by a
combination of abstract, intelligible principles and concrete, brute
categorical facts. Talk of the ‘generation’ of limits on metaphysical
possibility cannot be understood as involving matters ‘beyond’ the limits of
metaphysical possibility, because nothing lies beyond these limits: but having
recognized the orthogonal ‘as-actual’ and ‘as-counterfactual’ dimensions of
modal perspective, we can make sense of the limits of the ‘as-counterfactual’
covarying with options for the ‘as-actual’.

Phenomena supportive of bounded naturalism are widespread in metaphysics,
ranging from the analysis of law and chance, to the comprehension of
individual essence, to the determination of counterfactual truth, to the
ontogenesis of the domain of individuals and propositions. Literature on such
matters has long been stuck in dialectics between revision of intuitive
judgement about these phenomena and revision of intuitive logic. RMM promises
to unstick these many dialectics with a single, simple appeal to the
well-understood and widely recognized phenomenon of modal perspective.
Development of RMM thus promises to contribute significantly to the
enhancement of theory in contemporary metaphysics. 

A secondary outcome---equally significant as regards knowledge creation---will
be a deeper understanding of a cluster of interrelated issues in the
philosophy of language, including both (a: in 'descriptive semantics') a 
significant advance in our theory of conditionals and (b: in 'metasemantics’)
the articulation, modernization, and defense of an adequate framework for
formal theory of meaning, apt to represent both logical and pragmatic
phenomena in an intensional language. 

(a) A pair of independent ‘trivialization’ arguments (Gibbard’s and
Williamson’s) force a choice among (i) distinguishing the conditional from the
material conditional ‘not-p or q’ (after all, the latter is entailed by q and
by not-p, but ‘if p, q’ intuitively is not); (ii) a semantic analysis of ‘if
p’ as a standard intensional operator; (iii) intuitive data: to wit,
conformity to ‘conditional proof’, or the order-indifference of multiple
antecedents. Our best theory of the conditional (Stalnaker’s) preserves (i)
and (ii) at the expense of (iii); while the theory is serviceable and robust,
this choice stands in the way of a full understanding of conditional
reasoning. By explaining how we might instead preserve (i) and (iii) at the
expense of (ii), the proposed research promises a significant advance on our
understanding of an important area of language and cognition.

(b) The manoeuvre required to abandon (ii) displaces the action Stalnaker
proposes for ‘if p’ to a ‘postsemantic’ stage, posterior to the ‘semantic’
stage in which a meaning is composed for a full sentence, but prior to the
‘pragmatic’ stage in which context-sensitive expressions are evaluated. Making
room for this intermediate stage (let alone giving it independent motivation)
requires considerable delicacy in the articulation of a general framework in
theory of meaning. While Lewis’s 1980 ‘Index, context, and content’ advances
and defends such a framework, a more explicit and modernized presentation and
defense will be required for widespread recognition of its details and of the
power concealed within it.

Finally, the development of RMM promises significant opportunities for student training
and skill development at the graduate level, at two Canadian institutions.
These opportunities will positively impact graduate recruitment at both the
Manitoba and Toronto departments of philosophy, and will further contribute to
Toronto's position as a leading centre for the investigation of
perspective-relativity in language, logic, and metaphysics. 


